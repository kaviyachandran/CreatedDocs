%------------------------------------------------------------------------------------------
% PACKAGES AND OTHExsssR DOCUMEffdfdNT CONFIGURATIONS
%------------------------------------------------------------------------------------------
\documentclass[12pt]{article}
\usepackage{mathpazo}
%\usepackage{merriweather}
%\usepackage{libertine}
%\usepackage{libertinust1math}
%\usepackage[defaultsans]{droidsans}
%\renewcommand*\familydefault{\sfdefault}
%\usepackage[sfdefault,condensed]{roboto}
%\usepackage{cmbright}
%\usepackage[default,osfigures,scale=0.95]{opensans}
%\usepackage{tgpagella}
\usepackage[a4paper, total={6.2in, 9in}]{geometry}
\usepackage{url}
\usepackage{graphics}
\usepackage{graphicx}
\usepackage[T1]{fontenc}
\usepackage{amsmath}

\usepackage[pdftex]{hyperref}
%\usepackage[symbol]{footmisc}
\linespread{1.05}
\hypersetup{
   backref=section,
   bookmarksopen=true,
   bookmarksnumbered=true,
   bookmarksopenlevel=2,
   pdftitle={Master Thesis Proposal},
   pdfauthor={Mohammed Shameer Abubucker <mohammed.abubucker@smail.inf.h-brs.de>},
   pdfkeywords= {motion planning, humanoids, HRP4},
   %citecolor=blue,
   %citebordercolor={0 0.4 0.7}
   colorlinks,
   citecolor=blue,
   filecolor=black,
   linkcolor=black,
   urlcolor=black
}
\newcommand{\myspace}{\\\\}
\newcommand{\HRule}{\rule{\linewidth}{0.6mm}} 
\begin{document}
\begin{titlepage}

% Defines vjhvjva new command for
% the horizontal lines, change thickness here

\center % Center everything on thecf page
 
%----------------------------------------------------------------------------------------
%	HEADING SECTIONS
%----------------------------------------------------------------------------------------

\Large{Reference Document}\\[1.5cm] 
%----------------------------------------------------------------------------------------
%	TITLE SECTION
%----------------------------------------------------------------------------------------
\HRule
% \HRule \\[0.4cm]
{ \Large \bfseries Learning High-Level Environment Dynamics}\\\HRule \\[0.4cm]
% Title of your document
% \HRule
% \\[1.5cm]

\vspace*{1.5\baselineskip}
 
\normalsize{\bfseries Kaviya Dhanabalachandran
\footnote{\href{kaviya.dhanabalachandran@smail.inf.h-brs.de}{kaviya.dhanabalachandran@smail.inf.h-brs.de}}}\\[0.25cm]
% Minor heading such as course title
{\normalsize Master of Autonomous Systems}\\[0.5cm] % Minor heading such as
% course title
{\small Bonn-Rhein-Sieg University of Applied Sciences, Sankt
Augustin, Germany.}\\[0.05cm]
\&
\\[0.05cm]
% Minor heading such as course title
{\small Robert Bosch GmbH,\\ Robert-Bosch-Campus 1, 71272,\\ Renningen, Germany.}\\[0.7cm]
\small{\today}\\[1,5cm]
%April 24th, 2015
%----------------------------------------------------------------------------------------
%	AUTHOR SECTION
%----------------------------------------------------------------------------------------
%\vspace*{6\baselineskip} % Whitespace between location/year and editors


%\end{minipage}

\end{titlepage}
%%%%%%%%
%%%%%%%
\section{Objective}
\begin{itemize}
\item To estimate the time required to traverse an edge from sparse data.
\end{itemize}
%%%%%%%
%%%%%%%
\section{Approach}
There are two techniques to estimate unknown parameters from the observed data.
\begin{itemize}
\item Expectation Maximization
\item Bayesian Inference
\end{itemize}

First approach gives a point estimate of the unknown parameters while the bayesian inference gives a distribution of the estimated paramters. 
\subsection{Bayesian Parameter Estimation}
%%%%%%%
%%%%%%%
\subsection{Expectation Maximization}




%%%%%%%
\vspace*{6\baselineskip}
\newpage
\bibliography{referenceDoc}
\bibliographystyle{unsrt}

\end{document}